\documentclass{article}

% Language setting
% Replace `english' with e.g. `spanish' to change the document language
\usepackage[russian]{babel}
\usepackage[utf8]{inputenc}
\usepackage[T1]{fontenc}
\usepackage{mathtools}
\usepackage[pdf]{graphviz}

%%% Helper code for Overleaf's build system to
%%% automatically update output drawings when
%%% code in a \digraph{...} is modified
\usepackage{xpatch}
\makeatletter
\newcommand*{\addFileDependency}[1]{% argument=file name and extension
\typeout{(#1)}
\@addtofilelist{#1}
\IfFileExists{#1}{}{\typeout{No file #1.}}
}
\makeatother
\xpretocmd{\digraph}{\addFileDependency{#2.dot}}{}{}
% Set page size and margins
% Replace `letterpaper' with `a4paper' for UK/EU standard size
\usepackage[letterpaper,top=2cm,bottom=2cm,left=3cm,right=3cm,marginparwidth=1.75cm]{geometry}

% Useful packages
\usepackage{amsmath}
\usepackage{graphicx}
\usepackage[colorlinks=true, allcolors=blue]{hyperref}

\title{Теоретические модели вычислений \\
        ДЗ №2}
\author{А-13а-19 Сабитов Алексей}
\begin{document}

\maketitle

\section{Для следующих языков постройте КС-грамматику}
\subsection{$L = \{\omega \in \Sigma* |$  w содержит подстроку $aa\}$}
 
$$ S \Rightarrow aS \:|\: bS \:|\: cS \:|\: aaT $$
$$ T \Rightarrow aT \:|\: bT \:|\: cT \:|\: \lambda$$
 
\subsection{$l = \{\omega \in \Sigma* |$  w не палиндром\}}
$$S \Rightarrow aS(a|b|c)\; | \; bS(a|b|c)\; | \;cS(a|b|c) \; | \; aTb \; | \; aTc\; | \; bTa\; | \;bTc \; | \;cTa \; | \;cTb \; $$
$$ T \Rightarrow \; a \; | \; b \; | \; c \; |\;aT \; | \; bT \; | \; cT \; | \; |\; \lambda$$
\subsection{Алфавит: $\sum \{ \emptyset, N, `\{`,`\}`, \cup \} $ }
Постройте грамматику для языка $L = \{\omega \in  \sum^*|\omega $ - синтактически корректная строка, обозначающая множество $\}$  \newline
Грамматика: \newline
$S \Rightarrow A\; | \;A\: u\: S \;|\; \{A\: u\: S\}$\newline
$A \Rightarrow  B\; |\; \{\:\} \;|\; \{\:A\:\}\; |\; \{\:A,\:A\:\}$ \newline
$B \Rightarrow  C\:,\:B \;|\;C$\newline
$C \Rightarrow  N\:,\:O\;|\;O\:,\:N\;|\;N\;|\;O$\newline

\section{упражнение}
 
    В алфавите $\Omega = \{1, +, =\}$ мы можем записать выражения для суммы чисел $x + y = z$. Рассмотрим язык $A = \{1^m + 1^n = 1^{m+n} :\ | :\ m, n \in \mathbb{N}\}$
 
    \subsection{Докажите, что язык $A$ регулярный (построением) или нерегулярный (через лемму о накачке)}
 
        Будем доказывать, что язык нерегулярный:
 
        \begin{itemize}
            \item Фиксируем $n` = m + n + 2$
            \item Возьмем $w = 1^m + 1^{n+1} = 1^{m+n+1}$
            \item $|w| = 2(m + n) + 2 \geq n`$
            \item Рассмотрим разбиение:
 
            $x = \{1^m+\}$;    $y = \{1^{n+1}\}$
 
            $|xy| = m + n + 1 \leq n`; \: |y| = n + 1 \geq 1$
 
             $z = \{=1^{m+n+1}\}$
 
            \item $\forall k \geq 0: xy^kz \in L$ - не выполняется, так как при $k = 0 \;or\; k \geq 2 \Rightarrow \newline \Rightarrow  1^m + 1^{k(n+1)} = 1^{m+n+1} \Rightarrow m + kn + k \neq m + n + 1$.  \newline Следовательно, язык нерегулярный.
        \end{itemize}
 
    \subsection{Постройте КС-грамматику для языка $A$, показывающую, что $A$ - контекстно-свободный}
 
        \begin{itemize}
            \item $S \Rightarrow += \: | \:+1=1 \: | \: 1+=1  \: | \: 1+1T11 \: | \: 1S1$
 
            \item $T \Rightarrow = \: | \: 1T1 $
        \end{itemize}
        
        \section{упражнение}
 
    \subsection{С поводком}
 
        Пусть $D_1 = \{\omega \in \Omega^* \: | \: \omega$ описывает последовательность ваших шагов и шагов вашей собаки на прогулке с поводком $\}$.
 
        \begin{enumerate}
            \item Докажите, что язык $D_1$ регулярный (построением) или нерегулярный (через лемму о накачке)
 
            Построим ДКА, тем самым, покажем, что язык регулярный:
            \begin{center}
                \digraph{g}{
                    size="6,6";
                    rankdir="LR";
                    node [shape=point]; 0;
                    node [shape=circle]; 2 3 4 5;
                    node [shape=doublecircle]; 1;
                    0 -> 1;
                    1 -> 2 [label="h"];
                    2 -> 1 [label="d"];
                    2 -> 3 [label="h"];
                    3 -> 2 [label="d"];
                    1 -> 4 [label="d"];
                    4 -> 1 [label="h"];
                    4 -> 5 [label="d"];
                    5 -> 4 [label="h"];
                }
            \end{center}  
 
            \item Постройте КС-грамматику для языка $D_1$, показывающую, что $D_1$ - контекстно-свободный
 
                \begin{itemize}
                    \item $S \to hT \: | \: dR \: | \: \lambda$
 
                    \item $T \to hdT \: | \: dS$
 
                    \item $R \to dhR \: | \: hS$
                \end{itemize}
 
 
        \end{enumerate}
 
    \subsection{Без поводка}
 
        Пусть $D_2 = \{\omega \in \Omega^* \: | \: \omega$ описывает последовательность ваших шагов и шагов вашей собаки на прогулке без поводка $\}$.
 
        \begin{enumerate}
            \item Докажите, что язык $D_2$ регулярный (построением) или нерегулярный (через лемму о накачке)
 
            С помощью леммы о накачке покажем, что язык нерегулярный:
            \begin{itemize}
             
            
                \item Фиксируем $n`$
                \item Берем $w = h^nd^n$
                \item $|w| = 2n \geq n`$
                \item Рассмотрим разбиение:
 
                $x = h^i$
 
                $y = h^j$
 
                $|xy| = i + j = n; \: j > 0$
 
                $z = h^{n-i-j}d^n$
 
                \item $\forall k \geq 0: xy^kz \in L$ - не выполняется, так как при $k \geq 2\Rightarrow\newline \Rightarrow h^{i+kj+n-i-j}d^n \Rightarrow h^{n+j(k-1)}d^n$ \newlineЭто значит, что человек и собака не будут в одной точке.\newline Делаем вывод, что язык нерегулярный.
            \end{itemize}
 
            \item Постройте КС-грамматику для языка $D_2$, показывающую, что $D_2$ - контекстно-свободный
 
                \begin{itemize}
                    \item $S \to hSdS \: | \: dShS \: | \: \lambda$
                \end{itemize}
 
 
        \end{enumerate}
        \section{упражнение}
        \section{упражнение}
\subsection{ Привести алгоритм построения НКА по праволинейной грамматике. Доказать, что с помощью алгоритма мы можем получить только слова из языка грамматики. Проиллюстрировать алгоритм на грамматике:}
\newline
    $A \Rightarrow aB\;|\;bC$
 
    $B \Rightarrow aB\;|\;\lambda$
 
    $C \Rightarrow A\;|\;aD\;|\;bC$
 
    $D \Rightarrow aD\;|\;bD\;|\;\lambda$
\newline
{Алгоритм: }
\begin{itemize}
 \item {
    Множество   вершин   НКА   состоит   из   нетерминалов грамматики  и,  возможно, еще  одной  новой  вершины F, которая объявляется заключительной.
 }
 \item {
    Каждому  правилу  вида  A$\Rightarrow$aB  в  автомате  соответствует дуга  из  вершины А в  вершину В,  помеченная  символом а.  Каждому  правилу  вида  A$\Rightarrow$a  соответствует  дуга из  вершины А в  вершину F,  помеченная  символом а. Других дуг нет.
 }
 
 \item {
 Начальной вершиной   автомата   является   вершина, соответствующая    начальному    символу    грамматики. Заключительными    являются новая вершина F, если  она использовалась на шаге 2, и каждая вершина A, такая что для нетерминала A в грамматике есть правило  $A \Rightarrow \lambda$
 }
 
\end{itemize}
 
{Допустим, что наш алгоритм строит автомат, который допускает слова, которых нет в языке. Тогда существует  переход от одной нетерминальной вершины к другой, который не допускает язык.
 

 
  Следовательно, есть переход переход от одной нетерминальной вершины к другой, который не допускает язык $\Rightarrow$ должно было быть соответствующее правило, но его нет.
 
  Делаем вывод, что алгоритм допускает только слова из языка}
 
 \digraph{51}{
        size="6,6";
        rankdir="LR";
        node [shape=point]; 0;
        node [shape=circle]; A C;
        node [shape=doublecircle]; B D;
        0 -> A;
        A -> B [label="a"];
        B -> B [label="a"];
        A -> C [label="b"];
        C -> A;
        C -> D [label="a"];
        D -> D [label="a|b"];
    }
 
\subsection{ Привести алгоритм построения КС грамматики по НКА. Доказать, что с помощью алгоритма мы можем получить только слова из языка НКА. Проиллюстрировать алгоритм на грамматике:}\newline
{Алгоритм: }\newline
\begin{itemize}
 \item {
    Нетерминалами  грамматики  будут  вершины  автомата, терминалами — пометки дуг
 }
 \item {
    Для  каждой дуги  из  вершины А в  вершину В,  помеченная  символом а в грамматику  добавляется правило $A Rightarrow aB$. Для каждой заключительной вершины В в грамматику добавляется правило $B \Rightarrow \lambda$
 }
 
 \item {
Начальным  символом  будет  нетерминал,  соответствующий начальной  вершине.
 }
 
\end{itemize}
 
{Доказательство аналогично}
 
$$ Q_0 \Rightarrow aQ_0 | aQ_1 | Q_3 $$
$$ Q_1 \Rightarrow aQ_1 | aQ_2 | Q_2 | bQ_4 $$
$$ Q_2 \Rightarrow aQ_2 | bQ_2 | aQ_5 | \lambda$$
$$ Q_3 \Rightarrow bQ_0 | \lambda$$
$$ Q_4 \Rightarrow Q_5 | \lambda$$
$$ Q_5 \Rightarrow aQ_5 | bQ_2$$
\end{document}
