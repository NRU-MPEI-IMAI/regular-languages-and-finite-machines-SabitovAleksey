\documentclass[a4paper,12pt]{article}
 
\usepackage{cmap}		
\usepackage[utf8]{inputenc}			
\usepackage[english,russian]{babel}
\usepackage{framed}
\usepackage{hyperref}
\usepackage{amsmath}
\usepackage{graphicx}
\usepackage[colorinlistoftodos]{todonotes}
\usepackage{wrapfig}
\usepackage{lipsum}
\usepackage{color}
\usepackage{indentfirst}
\usepackage{times}
\usepackage{textcomp}
\usepackage{float}
\usepackage{listings}
\usepackage{xcolor}
\usepackage[T1]{fontenc}
 
\usepackage{morewrites}
 
\usepackage[pdf]{graphviz}
\usepackage{xpatch}
\makeatletter
\newcommand*{\addFileDependency}[1]{% argument=file name and extension
  \typeout{(#1)}
  \@addtofilelist{#1}
  \IfFileExists{#1}{}{\typeout{No file #1.}}
}
\makeatother
\xpretocmd{\digraph}{\addFileDependency{#2.dot}}{}{}
 
\usepackage[autosize]{dot2texi}
\usepackage{tikz}
\usetikzlibrary{shapes,arrows}

\title{Теоретические модели вычислений \\
        ДЗ №1: Регулярные языки и коненые автоматы}
\author{А-13а-19 Сабитов Алексей}

\begin{document}

\maketitle

\section{Построить конечный автомат, распознающий язык}
\subsection{ $L = \{\omega \in \{a,b,c\}^*| |\omega|_c = 1\}$}
    \begin{center}
    \digraph{t11}{
        size="6,6";
        rankdir="LR";
        0 [shape=point];
        s1 [shape=circle];
        s2 [shape=doublecircle];
        0 -> s1;
        s1 -> s1 [label="a, b"];
        s1 -> s2 [label="c"];
        s2 -> s2 [label="a, b"];
    }
    \end{center}
    \newpage
    \subsection{ $L2 = \{\omega \in \{a,b\}^* | |w|_a \leq 2, |\omega|_b \geq 2\}$} 
    Имеем 2 автомата:
    $|w|_a \leq 2$ 
    $L21 = \{\Sigma = \{a, b\}, Q_1 = \{1, 2, 3\},S_1 = \{ 1\}, T_1 = \{3\}, \delta_1 \}$
     \begin{center}
    \digraph{0120}{
        size="6,6";
        rankdir="LR";
        0 [shape=point];
        1 [shape=circle];
        2 [shape=circle];
        3 [shape=doublecircle];
        0 -> 1;
        1 -> 1 [label="a"];
        1 -> 2 [label="b"];
        2 -> 2 [label="a"];
        2 -> 3 [label="b"];
        3 -> 3 [label="a,b"];
    }
    \end{center}
    $\omega|_b \geq 2$ 
    $L22 = \{\Sigma = \{a, b\}, Q_2 = \{1, 2, 3\},S_2 =\{ 1\}, T_2 = \{1,2,3\}, \delta_2 \}$
        \begin{center}
    \digraph{t121}{
        size="6,6";
        rankdir="LR";
        0 [shape=point];
        1 [shape=doublecircle];
        2 [shape=doublecircle];
        3 [shape=doublecircle];
        0 -> 1;
        1 -> 1 [label="b"];
        1 -> 2 [label="a"];
        2 -> 2 [label="b"];
        2 -> 3 [label="a"];
        3 -> 3 [label="b"];
    }
    \end{center}
    \newpage
    Строим прямое произведение
        \begin{enumerate}
        \item $\Sigma = \{a, b\}$
        \item $Q = \{(1, 1), (1, 2), (1, 3), (2, 1), (2, 2), (2, 3), (3, 1), (3, 2), (3, 3)\}$
        \item $S = \{(1, 1)\}$
        \item $T = \{(1, 3), (2,3), (3, 3)\}$
        \item $\delta:$
 \begin{center}
      \begin{tabular}{ | c | c | c | c | }
\hline
$L21$ & $L22$ & a & b \\ \hline
 1 & 1 & 1,2 & 2,1 \\
 1 & 2 & 1,3 & 2,2 \\
 1 & 3 &     & 2,3 \\
 2 & 1 & 2,2 & 3,1 \\
 2 & 2 & 2,3 & 3,2 \\
 2 & 3 &     & 3,3 \\
 3 & 1 & 3,2 & 3,1 \\
 3 & 2 & 3,3 & 3,2 \\
 3 & 3 &     & 3,3 \\
\hline
\end{tabular} 
 \end{center}
    \end{enumerate}
    \digraph[scale=0.8]{t122}{
        size="6,6";
        rankdir="LR";
        0 [shape=point];
        node [shape=circle]; 11 12 21 22 31 32;
        node [shape=doublecircle]; 13 23 33;
        0 -> 11;
        11 -> 12 [label="b"];
        11 -> 21 [label="a"];
        12 -> 13 [label="b"];
        12 -> 22 [label="a"];
        13 -> 13 [label="b"];
        13 -> 23 [label="a"];
        21 -> 22 [label="b"];
        21 -> 31 [label="a"];
        22 -> 23 [label="b"];
        22 -> 32 [label="a"];
        23 -> 23 [label="b"];
        23 -> 33 [label="a"];
        31 -> 32 [label="b"];
        32 -> 33 [label="b"];
        33 -> 33 [label="b"];
    }
        \newpage
\subsection{$L = \{\omega \in \{a,b\}^*|\:|\omega|_a \neq |\omega|_b\}$}\newline
Данный язык подразумевает запоминание количество элементов, но КА "не умеет"   это делать.\newline
Следовательно, невозможно построить КА для данного языка.\newline
\subsection{$L = \{\omega \in \{a,b\}^*|\:\omega\omega=\omega\omega\omega\}$}\newline

    \digraph[scale=0.8]{t14}{
        size="6,6";
        rankdir="LR";
        node [shape=point]; 0;
        node [shape=doublecircle]; 1;
        0 -> 1 [label = "lambda"];
    } 
    
\section{Построить конечный автомат,используя прямое произведение }   
    \subsection{$L1 = \{\omega \in \{a, b\}^{*} \: | \: |\omega|_a \geq 2 \land |\omega|_b \geq 2\}$}
 
Имеем два автомата:\\ 
    $|\omega|_a \geq 2 $ \\ 
    $L11 = \{\Sigma = \{a, b\}, Q_1 = \{1, 2, 3\}, 1, T_1 = \{3\}, \delta_1 \}$
 
    \digraph[scale=0.8]{t211}{
        size="6,6";
        rankdir="LR";
        node [shape=point]; 0;
        node [shape=circle]; 1 2;
        node [shape=doublecircle]; 3;
        0 -> 1;
        1 -> 1 [label="b"];
        1 -> 2 [label="a"];
        2 -> 2 [label="b"];
        2 -> 3 [label="a"];
        3 -> 3 [label="a, b"];
    } \\
    $|\omega|_b \geq 2 $ \\ 
    $L12 = \{\Sigma = \{a, b\}, Q_2 = \{1, 2, 3\}, 1, T_2 = \{3\}, \delta_2 \}$
 
    \digraph[scale=0.8]{t212}{
        size="6,6";
        rankdir="LR";
        node [shape=point]; 0;
        node [shape=circle]; 1 2;
        node [shape=doublecircle]; 3;
        0 -> 1;
        1 -> 1 [label="a"];
        1 -> 2 [label="b"];
        2 -> 2 [label="a"];
        2 -> 3 [label="b"];
        3 -> 3 [label="a, b"];
    }
 
    \begin{enumerate}
        \item $\Sigma = \{a, b\}$
        \item $Q = \{(1, 1), (1, 2), (1, 3), (2, 1), (2, 2), (2, 3), (3, 1), (3, 2), (3, 3)\}$
        \item $S = \{(1, 1)\}$
        \item $T = \{(3, 3)\}$
        \item $\delta:$
 \begin{center}
    \begin{tabular}{|c|c|c|c|}
        \hline
        L11 & L12 & a & b  \\
        1 & 1 & 21 & 12  \\
        1 & 2 & 22 & 13  \\
        1 & 3 & 23 & 13  \\
        2 & 1 & 31 & 22  \\
        2 & 2 & 32 & 23  \\
        2 & 3 & 33 & 23  \\
        3 & 1 & 31 & 32  \\
        3 & 2 & 32 & 33  \\
        3 & 3 & 33 & 33  \\
    \end{tabular}
     \end{center}
    \end{enumerate}
 
    \begin{center}
        \digraph[scale=0.8]{t21}{
            size="6,6";
            rankdir="LR";
            node [shape=point]; 0;
            node [shape=circle]; 11 12 13 21 22 23 31 32;
            node [shape=doublecircle]; 33;
            0 -> 11;
            11 -> 21 [label="a"];
            11 -> 12 [label="b"];
            12 -> 22 [label="a"];
            12 -> 13 [label="b"];
            13 -> 23 [label="a"];
            13 -> 13 [label="b"];
            21 -> 31 [label="a"];
            21 -> 22 [label="b"];
            22 -> 32 [label="a"];
            22 -> 23 [label="b"];
            23 -> 33 [label="a"];
            23 -> 23 [label="b"];
            31 -> 31 [label="a"];
            31 -> 32 [label="b"];
            32 -> 33 [label="a, b"];
            33 -> 33 [label="a, b"];
        }
    \end{center}
    \subsection{$L2 = \{\omega \in \{a,b\}\: |\: |\omega| \geq 3 \land |\omega| \:  \text{нечетное} \}$}
    
    Имеем следующие автоматы:
    
    $|\omega| \geq 3$
    $L21 = \{\Sigma = \{a, b\}, Q_1 = \{1, 2, 3, 4\},S_1 =\{ 1\}, T_1 = \{4\}, \delta_1 \}$
    \begin{center}
    \digraph[]{t221}{
        size="6,6";
        rankdir="LR";
        node [shape=point]; 0;
        node [shape=circle]; 1 2,3;
        node [shape=doublecircle]; 4;
        0 -> 1;
        1 -> 2 [label="a,b"];
        2 -> 3 [label="a,b"];
        3 -> 4 [label="a,b"];
        4 -> 4 [label="a,b"];
    }
    \end{center}
    $|\omega| \: \text{нечетное}$
    $L22 = \{\Sigma = \{a, b\}, Q_1 = \{1, 2\},S_1 =\{ 1\}, T_1 = \{2\}, \delta_2 \}$
    \begin{center}
        \digraph[]{t222}{
            size="6,6";
            rankdir="LR";
            0 [shape=point];
            1 [shape=circle];
            2 [shape=doublecircle];
            0 -> 1;
            1 -> 2 [label="a,b"];
            2 -> 1 [label="a,b"];
        }
    \end{center}
Строим прямое произведение
        \begin{enumerate}
        \item $\Sigma = \{a, b\}$
        \item $Q = \{(1, 1), (1, 2), (2, 1), (2, 2), (3, 1), (3, 2), (4, 1), (4, 2), \}$
        \item $S = \{(1, 1)\}$
        \item $T = \{(4,2)\}$
        \item $\delta:$
 \begin{center}
      \begin{tabular}{ | c | c | c | c | }
\hline
$L21$ & $L22$ & a & b \\ \hline
 1 & 1 & 2,2 & 2,2 \\
 1 & 2 & 2,1 & 2,1 \\
 2 & 1 & 3,2 & 3,2 \\
 2 & 2 & 3,1 & 3,1 \\
 3 & 1 & 4,2 & 4,2 \\
 3 & 2 & 4,1 & 4,1 \\
 4 & 1 & 4,2 & 4,2 \\
 4 & 2 & 4,1 & 4,1 \\
\hline
\end{tabular} 
\end{center}
    \begin{center}
        \digraph[scale=0.8]{t22}{
            size="6,6";
            rankdir="LR";
            node [shape=point]; 0;
            node [shape=circle]; 11, 12, 21, 22, 31, 32, 41;
            node [shape=doublecircle]; 42;
            0 -> 11;
            11 -> 22 [label="a,b"];
            12 -> 21 [label="a,b"];
            21 -> 32 [label="a,b"];
            22 -> 31 [label="a,b"];
            31 -> 42 [label="a,b"];
            32 -> 41 [label="a,b"];
            41 -> 42 [label="a,b"];
            42 -> 41 [label="a,b"];
        }
    \end{center}
\end{enumerate}

\subsection{$L3 = \{\omega \in (a,b)| |\omega|_{a} \texttt{четно}\: \land |\omega|\: \texttt{кратно трем}\}$}

$|\omega|_{a} \: \text{четно} $

$L31= \{\Sigma = \{a, b\}, Q_1 = \{1, 2\},S_1 = \{1\}, T_1 = \{1\}, \delta_1 \}$

    \digraph{t231}{
        size="6,6";
        rankdir="LR";
        0 [shape=point];
        1 [shape=doublecircle];
        2 [shape=circle];
        0 -> 1;
        1 -> 1 [label="b"];
        1 -> 2 [label="a"];
        2 -> 2 [label="b"];
        2 -> 1 [label="a"];
    }
\\
$ |\omega|\: \texttt{кратно трем}$ 
$L32= \{\Sigma = \{a, b\}, Q_1 = \{1, 2, 3\}, S_2 = \{1\}, T_1 = \{1\}, \delta_1 \}$
\\
    \digraph{t232}{
        size="6,6";
        rankdir="LR";
        0 [shape=point];
        node [shape=circle]; 2 3;
        1 [shape=doublecircle];
        0 -> 1;
        1 -> 1 [label="a"];
        1 -> 2 [label="b"];
        2 -> 2 [label="a"];
        2 -> 3 [label="b"];
        3 -> 3 [label="a"];
        3 -> 1 [label="b"];
    }
\\
\begin{center}
    \begin{enumerate}
        \item $\Sigma = \{a, b\}$
        \item $Q = \{(1, 1), (1, 2), (1, 3), (2, 1), (2, 2), (2, 3)\}$
        \item $S = \{(1, 1)\}$
        \item $T = \{(1, 1)\}$
        \item $\delta:$
 \end{enumerate}
\begin{tabular}{ | c | c | c | c |}
\hline
L31 & L32 & a & b \\ \hline
1 & 1 & 21 & 12 \\
1 & 2 & 22 & 13 \\
1 & 3 & 23 & 11 \\
2 & 1 & 11 & 22 \\
2 & 2 & 12 & 23 \\
2 & 3 & 13 & 21 \\
\hline
\end{tabular}
\\
    \digraph[scale = 0.9]{t23}{
        size="6,6";
        rankdir="LR";
        0 [shape=point];
        11 [shape=doublecircle];
        node [shape=circle]; 12 13 21 22 23;
        0 -> 11;
        11 -> 21 [label="a"];
        11 -> 12 [label="b"];
        12 -> 22 [label="a"];
        12 -> 13 [label="b"];
        13 -> 23 [label="a"];
        13 -> 11 [label="b"];
        21 -> 11 [label="a"];
        21 -> 22 [label="b"];
        22 -> 12 [label="a"];
        22 -> 23 [label="b"];
        23 -> 13 [label="a"];
        23 -> 21 [label="b"];
    }
\end{center}

\newpage

    \subsection{$L4 = \overline{L3}$}
$\overline{L3} = \{\Sigma_3,Q_3, s_3, Q_3\backslash T_3, \delta_3 \}$ 
$Q_3\backslash T_3 = 12, 13, 21, 22, 23$ \\ 
\digraph{t24}{
    size="6,6";
    rankdir="LR";
    0 [shape=point];
    11 [shape=circle];
    node [shape=doublecircle]; 12 13 21 22 23;
    0 -> 11;
    11 -> 21 [label="a"];
    11 -> 12 [label="b"];
    12 -> 22 [label="a"];
    12 -> 13 [label="b"];
    13 -> 23 [label="a"];
    13 -> 11 [label="b"];
    21 -> 11 [label="a"];
    21 -> 22 [label="b"];
    22 -> 12 [label="a"];
    22 -> 23 [label="b"];
    23 -> 13 [label="a"];
    23 -> 21 [label="b"];
}

\section{Построить минимальный ДКА по регулярному выражению } 
\subsection{$(ab+aba)^*a$}
Автомат с лямбда переходами:
\\
\digraph{t311}{
        size="6,6";
        rankdir="LR";
        0 [shape=point];
        node [shape=doublecircle]; 10;
        node [shape=circle];0 1 2 3 4 5 6 7 8 9;
        0 -> 1;
        1 -> 2 [label="lambda"];
        1 -> 5 [label="lambda"];
        1 -> 9 [label="lambda"];
        2 -> 3 [label="a"];
        3 -> 4 [label="b"];
        4 -> 9 [label="lambda"];
        5 -> 6 [label="a"];
        6 -> 7 [label="b"];
        7 -> 8 [label="a"];
        8 -> 9 [label="lambda"];
        9 -> 10 [label="a"];
        9 -> 1 [label="lambda"];
    }
Удаляем Лямда переходы и используем алгоритм Томсона для получения ДКА: \newline
\digraph{t31}{
        size="6,6";
        rankdir="LR";
        node [shape=point]; 0;
        node [shape=circle];1, 47;
        node [shape=doublecircle];3610, 36810;
        0 -> 1;
        1 -> 3610 [label="a"];
        3610 -> 47 [label="b"];
        47 -> 36810 [label="a"];
        36810 -> 47 [label="b"];
        36810 -> 3610 [label="a"];
    }
    \newpage
\subsection{$a(a(ab)^*b)^*(ab)^*$}
Сначала построим части автомата: \newline

\hspace{0.4 cm}$a(a(ab)^*b)^*$ \newline
\digraph{t321}{
    size="6,6";
    rankdir="LR";
    0 [shape=point];
    2 [shape=doublecircle];
    node [shape=circle]; 1 3 4;
    0 -> 1;
    1 -> 2 [label="a"];
    2 -> 3 [label="a"];
    3 -> 4 [label="a"];
    4 -> 3 [label="b"];
    3 -> 2 [label="b"];
}
\hspace{0.3 cm} 2 - ая часть автомата \newline

\hspace{0.4 cm}$(ab)^*$ \newline
\digraph{t322}{
    size="6,6";
    rankdir="LR";
    0 [shape=point];
    5 [shape=doublecircle];
    6 [shape=circle];
    0 -> 5;
    5 -> 6 [label = "a"];
    6 -> 5 [label = "b"];
} \newpage
Соединяем ( с помощью лямбда-перехода ): \newline
\digraph{t320}{
    size="6,6";
    rankdir="LR";
    0 [shape=point];
    node [shape=doublecircle]; 2 5;
    node [shape=circle]; 1 3 4 6;
    0 -> 1;
    1 -> 2 [label="a"];
    2 -> 3 [label="a"];
    3 -> 4 [label="a"];
    4 -> 3 [label="b"];
    3 -> 2 [label="b"];
    5 -> 6 [label = "a"];
    6 -> 5 [label = "b"];
    2 -> 5 [label = "lambda"];
}
\newline
\hspace{0.3 cm}Удаляем лямбда-переходы - получаем ДКА: \\\newline
\digraph{t32}{
    size="6,6";
    rankdir="LR";
    0 [shape=point];
    2 [shape=doublecircle];
    node [shape=circle]; 1 3 4;
    0 -> 1;
    1 -> 2 [label="a"];
    2 -> 3 [label="a"];
    3 -> 4 [label="a"];
    4 -> 3 [label="b"];
    3 -> 2 [label="b"];
}
\newpage
\subsection{$(a + (a+b)(a + b)b)^*$} \newline

Сначала строим НКА: \newline
    \begin{center}
        \digraph{t330}{
            size="6,6";
            rankdir="LR";
            node [shape=point]; 0;
            node [shape=doublecircle]; 1;
            node [shape=circle]; 2 3;
            0 -> 1;
            1 -> 1 [label="a"];
            1 -> 2 [label="a, b"];
            2 -> 3 [label="a, b"];
            3 -> 1 [label="b"];
            }
        \end{center}
 
По НКА с помощью алгоритма Томсона строим ДКА: \newline
    \begin{center}
        \digraph[scale=0.7]{t33}{
            size="6,6";
            rankdir="LR";
            node [shape=point]; 0;
            node [shape=circle]; 2 3 23;
            node [shape=doublecircle]; 1 12 13 123;
            0 -> 1;
            1 -> 12 [label="a"];
            1 -> 2 [label="b"];
            2 -> 3 [label="a, b"];
            3 -> 1 [label="b"]
            12 -> 123 [label="a"];
            12 -> 23 [label="b"];
            123 -> 123 [label="a, b"];
            23 -> 3 [label="a"];
            23 -> 13 [label="b"];
            13 -> 12 [label="a, b"]
        }  
    \end{center}
    \newpage
\subsection{$(b+c)((ab)^*c + (ba)^*)^*$} \newline
Используя алгоритм Томсона, строим ДКА:\newline
	\digraph{t34}{
        size="6,6";
        rankdir="LR";
        node [shape=point]; 0;
        node [shape=doublecircle]; 1;
        node [shape=circle]; 2 3 4;
        0 -> 1;
        1 -> 1 [label="c"];
        1 -> 2 [label="a"];
        2 -> 3 [label="b"];
        3 -> 2 [label="a"];
        3 -> 1 [label="c"];
        1 -> 4 [label="b"]
        4 -> 1 [label="a"]
	}
\subsection{$(a+b)^*(aa+bb+abab+baba)(a+b)^*$} \newline
Построим НКА:\newline
\digraph[scale=0.6]{t351}{
                node [shape = point]; 0;
                node [shape = doublecircle]; 13;
                node [shape = circle]; 1 2 3 4 5 6 7 8 9 10 11 12;
                rankdir=LR; 
                0 -> 1;
                1 -> 2 [label = "a, b"];
                2 -> 1 [label = "a, b"];
                2 -> 3 [label = "a"];
                2 -> 8 [label = "b"];
                3 -> 4 [label = "a"];
                3 -> 5 [label = "b"];
                4 -> 13 [label = "a, b"];
                5 -> 6 [label = "a"];
                6 -> 7 [label = "b"];
                7 -> 13 [label = "a, b"];
                8 -> 9 [label = "b"];
                8 -> 10 [label = "a"];
                9 -> 13 [label = "a, b"];
                10 -> 11 [label = "b"];
                11 -> 12 [label = "a"];
                12 -> 13 [label = "a, b"];
                13 -> 13 [label = "a,b"];
            }\newline
\includegraphics[scale = 0.5]{пол.jpg}
\begin{center}
    
\end{center}
\section{Определить является ли язык регулярным или нет}
\subsection{$L = \{(aab)^nb(aba)^m\:|n \geq 0,\:m \geq 0\}$}\newline
Этот язык является регулярным, строим автомат:\newline
\digraph[scale = 0.9]{t41}{
    size="6,6";
    rankdir="LR";
    node [shape=point]; 0;
    node [shape=circle]; 1 2 3 4 6 7;
    node [shape=doublecircle]; 5 8;
    0 -> 1;
    1 -> 2 [label="a"];
    2 -> 3 [label="a"];
    3 -> 4 [label="b"];
    4 -> 2 [label="a"];
    4 -> 5 [label="b"];
    5 -> 6 [label="a"]
    6 -> 7 [label="b"]
    7 -> 8 [label="a"];
    8 -> 6 [label="a"]
}
\subsection{$L = \{uaav\:|\:u \in\{a,b\},v \in\{a,b\},|u|_b \geq|v|_a\}$}\newline
Нужно разбить язык, как $uxyzv$, причем $|xy| \leq n$, $|y| \geq 1$.\newline
И слово $uxy^izv$ при $\forall i \geq 1$ пренадлежит языку $L$.\newline
Берем отрицание языка $L$:\newline
$$\overline{L} = \{uaav\:|\:u \in\{a,b\},v \in\{a,b\},|u|_b <|v|_a\} $$\newline
Фиксируем произвольное $n\in N$.\newline
Берем слово:\newline
$$b^naaa^n$$ 
Очевидно, длина слова не меньше n: $|w|=2n+2\geq n$.\newline
Разобьем язык следующим образом:\newline
$$b^{n-l}b^la^{2}a^n$$\newline
где $\:x=b^{n-l};\:y = b^l;\:z = a^{n+2};$ и  $ l \geq 1;\:l \leq n; $\newline
При достаточном большом $i$ (накачиваем $y$) слово $xy^iz = b^{n-l}(b^{l})^ia^{2}a^n $ не будет пренадлежать языку $\overline{L}$, следовтельно, язык  $\overline{L}$ - нерегулярный, а, значит, язык $L$ тоже не регулярный.\newline
ч.т.д.
\subsection{$L=\{a^m\omega | \omega \in \{a,b\}^*, 1\leq |\omega|_b\leq m\}$}
Рассмотрим отрицание языка L:\newline $$\overline{L}=\{a^m\omega | \omega \in \{a,b\}^*, |\omega|_b\geq m\}$$
\newline
Фиксируем произвольное $n\in N$. \newline Берем слово:\newline $$w = a^nb^n$$.
\newline
Очевидно, длина слова не меньше n: $|w|=2n\geq n$.
\newline
Разобьем язык следующим образом:\newline
    $$a^{l_1}a^{l_2}a^{n-l_1-l_2}$$ 
    
    где $x=a^{l_1}$;\:$y=a^{l_2}$;\: $z=a^{n-l_1-l_2}b^n$ и $l_2 \geq 1;\:l_1 + l_2 \leq n;$\newline
При достаточном большом $i$ (накачиваем $y$) слово $xy^iz = a^{l_1}(a^{l_2})^ia^{n-l_1-l_2}b^n $ не будет пренадлежать языку $\overline{L}$, следовательно, язык  $\overline{L}$ - нерегулярный, а, значит, язык $L$ тоже не регулярный.\newline
ч.т.д.
\subsection{$L=\{a^kb^ma^n | k = n\vee m > 0\}$}
Фиксируем произвольное $n\in N$.\newline Берем слово\newline  $$w = a^{n-1}ba^n\in L$$\newline
Очевидно, длина слова не меньше n: $|w|=2n\geq n$.
\newline
Рассмотрим следующее разбиение слова:
$$ a^{n-1-l}a^lba^n$$
где $x=a^{n-1-l}$;\:$y=a^{l}b$;\: $z=a^{n}$ и $l \geq 0;\:l \leq n-1;$\newline
Других разбиений нет.\newline
При $i = 0$ слово $xy^iz = a^{n-1-l}(a^{l}b)^ia^{n} = a^{n-1-l}a^n  $ не будет пренадлежать языку $L$, следовательно, язык  $L$ - нерегулярный.\newline
ч.т.д.
\subsection{$L=\{ucv|\: u \in \{a,b\}^*\:v \in \{a,b\}^* u \neq v^R\}$}
Фиксируем произвольное $n\in N$.\newline Берем слово\newline  $$w =a^nca^{2n}\in L$$\newline
Очевидно, длина слова не меньше n: $|w|=3n+1\geq n$.
\newline
Рассмотрим следующее разбиение слова:
$$ a^{n-l}a^lca^{2n}$$
где $x=a^{n-l}$;\:$y=a^{l}$;\: $z=ca^{2n}$ и $l > 0;\:l \leq n;$\newline
Других разбиений нет.\newline
При $i = 2$ слово принимает такой вид: $xy^iz = a^{n-l}(a^{l})^ica^{2n} = a^{n-l+il}ca^n = a^{n+l}ca^{2n}  $;\newlineТаким образом, при $l = n$:\newline
$$a^{2n}ca^{2n}$$То есть, $u = v^R $, следовательно, слово не будет пренадлежать языку $L$, следовательно, язык  $L$ - нерегулярный.\newline
ч.т.д.
\end{document}


